\documentclass[12pt]{article}
 
\usepackage[margin=1in]{geometry} 
\usepackage{amsmath,amsthm,amssymb}
\usepackage{karnaugh-map}
\usepackage{circuitikz}
\usepackage{tikz}
\usetikzlibrary{shapes.gates.logic.US}
\usetikzlibrary{circuits.ee.IEC}
 
\begin{document}
\title{EE5803 - FPGA LAB \\ Assignment-2}
\author{Venkatesh Parvathala \\ EE22RESCH01005}
 
\maketitle
\textbf{Q}. Verify the result obtained in Assignment-1 using Arduino. Given problem is
\begin{equation}
    F(X,Y,Z,W)=\sum(0,1,2,3,4,5,10,11,14)
\end{equation}

\textbf{Sol}. The given boolean expression can be expressed in K-Map as follows,

\begin{figure}[h]
    \centering
    \begin{karnaugh-map}[4][4][1][][]
    \maxterms{6,7,8,9,12,13,15}
    \minterms{0,1,2,3,4,5,10,11,14}
    \implicant{0}{5}
    \implicantedge{3}{2}{11}{10}
    \implicant{14}{10}
    \draw[color=black, ultra thin] (0, 4) --
    node [pos=0.7, above right, anchor=south west] {$ZW$} % Y label
    node [pos=0.7, below left, anchor=north east] {$XY$} % X label
    ++(135:1);
        
\end{karnaugh-map}
\end{figure}


The implicants in 0,1,4,5 gives us $\bar{X}\bar{Z}$

The implicatns in 2,3,10,11 gives us $\bar{Y}Z$

The implicants in 10,14 gives us $XZ\bar{W}$ \\

Combining all the above terms will give us 
\begin{equation}
    F(X,Y,Z,W) = \bar{X}\bar{Z} + \bar{Y}Z + XZ\bar{W}
\end{equation}


The above boolean expression which is same as obtained in Assignment-1 is verified using the following code in Arduino.
\newpage

The corresponding code to be flashed into Arduino is as follows,
\begin{verbatim}
#include <Arduino.h>

# define X 2
# define Y 3
# define Z 4
# define W 5

int x, y, z, w, out;

void setup() {
  pinMode(LED_BUILTIN,OUTPUT);
  pinMode(X, INPUT);
  pinMode(Y, INPUT);
  pinMode(Z, INPUT);
  pinMode(W, INPUT);
  // put your setup code here, to run once:
}

void loop() {
	x = digitalRead(X);
	y = digitalRead(Y);
	z = digitalRead(Z);
	w = digitalRead(W);

	out = ((!x)&&(!z)) | ((!y)&&(z)) | ((x)&&(z)&&(!w)); 
	if (out==1)
		digitalWrite(LED_BUILTIN, HIGH);
	else
		digitalWrite(LED_BUILTIN, LOW);
}
\end{verbatim}

The above code can be found in Git with the name Assignment-2.cpp and the same is verified by flashing into the Arduino.
\end{document}
