\documentclass[12pt]{article}
 
\usepackage[margin=1in]{geometry} 
\usepackage{amsmath,amsthm,amssymb}
\usepackage{karnaugh-map}
\usepackage{circuitikz}
\usepackage{tikz}
\usetikzlibrary{shapes.gates.logic.US}
\usetikzlibrary{circuits.ee.IEC}

\newcommand{\N}{\mathbb{N}}
\newcommand{\Z}{\mathbb{Z}}
 
\begin{document}
\title{EE5803 - FPGA LAB \\ Assignment-1}
\author{Venkatesh Parvathala \\ EE22RESCH01005}
 
\maketitle
\textbf{Q}. Reduce the following boolean expression to its simplest form using K-Map.
\begin{equation}
    F(X,Y,Z,W)=\sum(0,1,2,3,4,5,10,11,14)
\end{equation}

\textbf{Sol}. First we will build a Truth Table for the given expression as below,

\begin{table}[h]
    \centering
    \begin{tabular}{|c|c|c|c|c|}
    \hline
    X & Y & Z & W & \textbf{F} \\ \hline
    0 & 0 & 0 & 0 & \textbf{1}\\
    0 & 0 & 0 & 1 & \textbf{1}\\
    0 & 0 & 1 & 0 & \textbf{1}\\
    0 & 0 & 1 & 1 & \textbf{1}\\
    0 & 1 & 0 & 0 & \textbf{1}\\
    0 & 1 & 0 & 1 & \textbf{1}\\
    0 & 1 & 1 & 0 & \textbf{0}\\
    0 & 1 & 1 & 1 & \textbf{0}\\
    1 & 0 & 0 & 0 & \textbf{0}\\
    1 & 0 & 0 & 1 & \textbf{0}\\
    1 & 0 & 1 & 0 & \textbf{1}\\
    1 & 0 & 1 & 1 & \textbf{1}\\
    1 & 1 & 0 & 0 & \textbf{0}\\
    1 & 1 & 0 & 1 & \textbf{0}\\
    1 & 1 & 1 & 0 & \textbf{1}\\
    1 & 1 & 1 & 1 & \textbf{0}\\ \hline
    \end{tabular}
    \caption{The Truth Table}
\end{table}

\newpage
K-Map for the given expression as below

\begin{center}
\begin{karnaugh-map}[4][4][1][][]
    \maxterms{6,7,8,9,12,13,15}
    \minterms{0,1,2,3,4,5,10,11,14}
    \implicant{0}{5}
    \implicantedge{3}{2}{11}{10}
    \implicant{14}{10}
    \draw[color=black, ultra thin] (0, 4) --
    node [pos=0.7, above right, anchor=south west] {$ZW$} % Y label
    node [pos=0.7, below left, anchor=north east] {$XY$} % X label
    ++(135:1);
        
    \end{karnaugh-map}   
\end{center}


The implicants in 0,1,4,5 gives us $\bar{X}\bar{Z}$

The implicatns in 2,3,10,11 gives us $\bar{Y}Z$

The implicants in 10,14 gives us $XZ\bar{W}$ \\

Combining all the above terms will give us 
\begin{equation}
    F(X,Y,Z,W) = \bar{X}\bar{Z} + \bar{Y}Z + XZ\bar{W}
\end{equation}

In order to implement it using NAND gates, we will write the above SOP form as below
\begin{equation}
    F(X,Y,Z,W) = \overline{(\overline{\bar{X}\bar{Z} + \bar{Y}Z + XZ\bar{W}})}
\end{equation}

\begin{equation}
    F(X,Y,Z,W) = \overline{(\overline{\bar{X}\bar{Z}} . \overline{\bar{Y}Z} . \overline{XZ\bar{W}})}
\end{equation}


The above equation can be implemented using NAND gates and the corresponding code is available at \textit{./assignment\_1.c}

\end{document}
