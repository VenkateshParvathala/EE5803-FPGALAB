\documentclass[12pt]{article}
 
\usepackage[margin=1in]{geometry} 
\usepackage{amsmath,amsthm,amssymb}
\usepackage{karnaugh-map}
\usepackage{circuitikz}
\usepackage{tikz}
\usetikzlibrary{shapes.gates.logic.US}
\usetikzlibrary{circuits.ee.IEC}
 
\begin{document}
\title{EE5803 - FPGA LAB \\ Assignment-1}
\author{Venkatesh Parvathala \\ EE22RESCH01005}
 
\maketitle
\textbf{Q}. Reduce the following boolean expression to its simplest form using K-Map.
\begin{equation}
    F(X,Y,Z,W)=\sum(0,1,2,3,4,5,10,11,14)
\end{equation}

\textbf{Sol}. First we will build a Truth Table for the given expression as below,

\begin{table}[h]
\centering
\begin{tabular}{|c|c|c|c|c|}
\hline
X & Y & Z & W & \textbf{F} \\ \hline
0 & 0 & 0 & 0 & \textbf{1}\\
0 & 0 & 0 & 1 & \textbf{1}\\
0 & 0 & 1 & 0 & \textbf{1}\\
0 & 0 & 1 & 1 & \textbf{1}\\
0 & 1 & 0 & 0 & \textbf{1}\\
0 & 1 & 0 & 1 & \textbf{1}\\
0 & 1 & 1 & 0 & \textbf{0}\\
0 & 1 & 1 & 1 & \textbf{0}\\
1 & 0 & 0 & 0 & \textbf{0}\\
1 & 0 & 0 & 1 & \textbf{0}\\
1 & 0 & 1 & 0 & \textbf{1}\\
1 & 0 & 1 & 1 & \textbf{1}\\
1 & 1 & 0 & 0 & \textbf{0}\\
1 & 1 & 0 & 1 & \textbf{0}\\
1 & 1 & 1 & 0 & \textbf{1}\\
1 & 1 & 1 & 1 & \textbf{0}\\ \hline
\end{tabular}
\caption{Truth Table}
\end{table}


\newpage
We can express the same boolean expression in K-Map as follows

\begin{figure}[h]
    \centering
    \begin{karnaugh-map}[4][4][1][][]
    \maxterms{6,7,8,9,12,13,15}
    \minterms{0,1,2,3,4,5,10,11,14}
    \implicant{0}{5}
    \implicantedge{3}{2}{11}{10}
    \implicant{14}{10}
    \draw[color=black, ultra thin] (0, 4) --
    node [pos=0.7, above right, anchor=south west] {$ZW$} % Y label
    node [pos=0.7, below left, anchor=north east] {$XY$} % X label
    ++(135:1);
        
\end{karnaugh-map}
\end{figure}


The implicants in 0,1,4,5 gives us $\bar{X}\bar{Z}$

The implicatns in 2,3,10,11 gives us $\bar{Y}Z$

The implicants in 10,14 gives us $XZ\bar{W}$ \\

Combining all the above terms will give us 
\begin{equation}
    F(X,Y,Z,W) = \bar{X}\bar{Z} + \bar{Y}Z + XZ\bar{W}
\end{equation}

In order to implement it using NAND gates, we will write the above SOP form as below
\begin{equation}
    F(X,Y,Z,W) = \overline{(\overline{\bar{X}\bar{Z} + \bar{Y}Z + XZ\bar{W}})}
\end{equation}

\begin{equation}
    F(X,Y,Z,W) = \overline{(\overline{\bar{X}\bar{Z}} . \overline{\bar{Y}Z} . \overline{XZ\bar{W}})}
\end{equation}
\newpage

The corresponding circuit diagram with only NAND gates can be drawn as follows.
\begin{figure}[h]
\centering
\begin{circuitikz}[label distance=2mm, scale=2,
  connection/.style={draw,circle,fill=black,inner sep=1.5pt}
  ]
\node (x) at (0.5,0) {$X$};
\node (y) at (1.5,0) {$Y$};
\node (z) at (2.5,0) {$Z$};
\node (w) at (3.5,0) {$W$};
\node[nand gate US, draw, rotate=-90, logic gate inputs=nn, scale=1.5] at ($(x)+(0.5,-1)$) (t01) {};
\node[nand gate US, draw, rotate=-90, logic gate inputs=nn, scale=1.5] at ($(y)+(0.5,-1)$) (t02) {};
\node[nand gate US, draw, rotate=-90, logic gate inputs=nn, scale=1.5] at ($(z)+(0.5,-1)$) (t03) {};
\node[nand gate US, draw, rotate=-90, logic gate inputs=nn, scale=1.5] at ($(w)+(0.5,-1)$) (t04) {};
\node[nand gate US, draw, rotate=0, logic gate inputs=nn, scale=1.5] at ($(w)+(1.5,-2)$) (t1) {};
\node[nand gate US, draw, rotate=0, logic gate inputs=nn, scale=1.5] at ($(w)+(1.5,-3)$) (t2) {};
\node[nand gate US, draw, rotate=0, logic gate inputs=nnn, scale=1.5] at ($(w)+(1.5,-4)$) (t3) {};
\node[nand gate US, draw, logic gate inputs=nnn, scale=1.25] at ($(t3.output) + (1, 1)$) (orTot) {};
\node (im1) at ($(t1.output)+(0.2,0.2)$) {$\bar{X}\bar{Z}$};
\node (im2) at ($(t2.output)+(0.2,0.2)$) {$\bar{Y}Z$};
\node (im3) at ($(t3.output)+(0.2,-0.2)$) {$XZ\bar{W}$};
\draw (x) -- ($(x) + (0,-4.5)$);
\draw (y) -- ($(y) + (0,-4.5)$);
\draw (z) -- ($(z) + (0,-4.5)$);
\draw (w) -- ($(w) + (0,-4.5)$);
\draw (0.5,-0.5) -| (t01.input 1) ;
\draw (0.5,-0.5) -| (t01.input 2) ;
\draw (1.5,-0.5) -| (t02.input 1) ;
\draw (1.5,-0.5) -| (t02.input 2) ;
\draw (2.5,-0.5) -| (t03.input 1) ;
\draw (2.5,-0.5) -| (t03.input 2) ;
\draw (3.5,-0.5) -| (t04.input 1) ;
\draw (3.5,-0.5) -| (t04.input 2) ;
\draw (t01.output) -- ($(x) + (0.5,-4.5)$);
\draw (t02.output) -- ($(y) + (0.5,-4.5)$);
\draw (t03.output) -- ($(z) + (0.5,-4.5)$);
\draw (t04.output) -- ($(w) + (0.5,-4.5)$);
\draw (t01.output) |- (t1.input 1) node[connection,pos=0.5]{};
\draw (t03.output) |- (t1.input 2) node[connection,pos=0.5]{};
\draw (t02.output) |- (t2.input 1) node[connection,pos=0.5]{};
\draw (z) |- (t2.input 2) node[connection,pos=0.5]{};
\draw (x) |- (t3.input 1) node[connection,pos=0.5]{};
\draw (z) |- (t3.input 2) node[connection,pos=0.5]{};
\draw (t04.output) |- (t3.input 3) node[connection,pos=0.5]{};
\draw (t1.output) -- ([xshift=0.4cm]t1.output) |- (orTot.input 1);
\draw (t2.output) -- ([xshift=0.3cm]t2.output) |- (orTot.input 2);
\draw (t3.output) -- ([xshift=0.3cm]t3.output) |- (orTot.input 3);
\draw (orTot.output) -- node[above]{$F$} ($(orTot) + (1, 0)$);
\end{circuitikz}
\caption{Circuit Diagram using NAND gates}
\label{ckt1}
\end{figure}

The above logic circuit is realized using a C-program and the corresponding code is available at \textit{./assignment\_1.c}
\end{document}
